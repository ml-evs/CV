\documentclass[11pt,a4paper,sans]{moderncv}        % possible options include font size ('10pt', '11pt' and '12pt'), paper size ('a4paper', 'letterpaper', 'a5paper', 'legalpaper', 'executivepaper' and 'landscape') and font family ('sans' and 'roman')

%\usepackage{libertine}
%\usepackage[scaled=0.9]{inconsolata}
\usepackage{fontspec}
\setsansfont[BoldFont={Archivo Narrow Bold}]{Archivo Narrow}
\setmonofont[Scale=0.9, BoldFont={Iosevka Term Medium}]{Iosevka Term Medium}
% moderncv themes
\moderncvstyle{classic}                             % style options are 'casual' (default), 'classic', 'banking', 'oldstyle' and 'fancy'
\moderncvcolor{blue}                               % color options 'black', 'blue' (default), 'burgundy', 'green', 'grey', 'orange', 'purple' and 'red'
\definecolor{color2}{RGB}{12, 69, 129}
\definecolor{darkgray}{gray}{0.4}
\usepackage[unicode,pdfencoding=auto]{hyperref}
%\definecolor{color2}{RGB}{56,115,178}
%\usepackage{doi}
\usepackage[style=nature,sorting=sortyear,defernumbers=true]{biblatex}
\addbibresource{pubs.bib}

\DeclareSortingScheme{sortyear}{
    \sort[direction=ascending]{
        \field{sortyear}
    }
}

\DeclareSourcemap{
  \maps[datatype=bibtex]{
    \map{
      \step[fieldsource=keywords,
            match=inprep,
            final]
      \step[fieldset=keywords, fieldvalue=inprep]
    }
  }
}

%hyphenation
\tolerance=1
\emergencystretch=\maxdimen
\hyphenpenalty=10000
\hbadness=10000

% Reverse numbering in publications list
\newcounter{entrycount}
\AtDataInput{\stepcounter{entrycount}}
\DeclareFieldFormat{labelnumber}{\mkrevbibnum{#1}}
\newcommand{\mkrevbibnum}[1]{\number\numexpr\value{entrycount}+1-#1}

\newcommand\colourhref[3][color2]{\href{#2}{\color{#1}#3}}
\newcommand{\doi}[1]{DOI: {\colourhref{https://dx.doi.org/#1}{\texttt{#1}}}}
%\renewcommand{\familydefault}{\sfdefault}         % to set the default font; use '\sfdefault' for the default sans serif font, '\rmdefault' for the default roman one, or any tex font name
\nopagenumbers{}                                  % uncomment to suppress automatic page numbering for CVs longer than one page

\cfoot{\emph{Last modified: \today}}

\usepackage{etoolbox}
\makeatletter
\patchcmd{\makecvtitle}% <cmd>
  {\httplink{\@homepage}}% <search>
  {{\ifx\@homepage@shorthand\relax
     \httplink{\@homepage}% Used \homepage{<URL>}
   \else
     \httplink[\@homepage@shorthand]{\@homepage}% Used \homepage[<desc>]{<URL>}
   \fi}}% <replace>
  {}{}% <succes><failure>
\patchcmd{\thebibliography}
{\advance\leftmargin\labelsep}
  {\labelsep=0.7cm \advance\leftmargin\labelsep}{}{}
\RenewDocumentCommand{\homepage}{o m}{%
  \let\@homepage@shorthand\relax%
  \providecommand\@homepage{#2}%
  \IfNoValueF{#1}{\def\@homepage@shorthand{#1}}%
}
\makeatother

% adjust the page margins
\usepackage[scale=0.8]{geometry}
%\setlength{\hintscolumnwidth}{3cm}                % if you want to change the width of the column with the dates
%\setlength{\makecvheadnamewidth}{12cm}            % for the 'classic' style, if you want to force the width allocated to your name and avoid line breaks. be careful though, the length is normally calculated to avoid any overlap with your personal info; use this at your own typographical risks...

% personal data
\name{Matthew}{Evans}
%\title{Resumé title}                               % optional, remove / comment the line if not wanted
\email{me388@cam.ac.uk}                               % optional, remove / comment the line if not wanted
\homepage{ml-evs.science}
\social[github]{ml-evs}                           % optional, remove / comment the line if not wanted
\quote{\texttt{energy storage \textbullet\,ab initio calculations\\ crystal structure databases \textbullet\, software development}}

\renewcommand*{\bibliographyitemlabel}{[\arabic{enumiv}]}

\begin{document}
%-----       resume       ---------------------------------------------------------
\pagestyle{empty}
\makecvtitle
\section{\texttt{EDUCATION}}
\cventry{2016--}{PhD Physics}{Theory of Condensed Matter Group}{University of Cambridge}{}{}  % arguments 3 to 6 can be left empty
\cventry{2015--2016}{MPhil Scientific Computing}{}{University of Cambridge}{\textit{Distinction}}{}  % arguments 3 to 6 can be left empty
\cventry{2011--2015}{MPhys Physics with Theoretical Physics}{}{University of Manchester}{\textit{First Class (Hons)}}{}

\section{\texttt{EXPERIENCE}}
\cvitem{PhD}{\textbf{Crystal structure prediction for next-generation energy storage applications}}
\cvitem{}{with Dr Andrew Morris \emph{(University of Cambridge/University of Birmingham)}}
\cvitem{}{Discovery and computational characterisation of novel high-capacity anode materials for Li-, Na- and K-ion batteries, using \emph{ab initio} random structure searching (AIRSS) and evolutionary approaches, implemented in the open-source \colourhref{http://ilustrado.readthedocs.io}{\texttt{ilustrado}} package. PhD research undertaken as a member of the EPSRC CDT for Computational Methods in Materials Science.}

\cvitem{Internship}{\textbf{Scientific Software Developer}}
\cvitem{}{Enthought Inc., Cambridge}
\cvitem{}{Worked on the open source, Horizon 2020 funded \colourhref{https://github.com/force-h2020}{FORCE project}, with a focus on adding functionality to the workflow manager for multi-criteria optimisations, including developing a Bayesian optimisation plugin. Helped develop Cython bindings for the ACADO toolkit.}

\cvitem{MPhil}{\textbf{High-throughput \emph{ab initio} materials discovery}}
\cvitem{}{with Dr Andrew Morris \emph{(University of Cambridge)}}
\cvitem{}{Database approaches to materials design; developed an open-source software package, \colourhref{http://matador.science}{\texttt{matador}}, to aggregate and {analyse} the results of first-principles calculations.}

\cvitem{MPhys}{\textbf{Electronic structure of defects in graphene superlattices}}
\cvitem{}{with Prof Francisco Guinea \emph{(University of Manchester)}}
%\cvitem{}{Nearly-free electron model of graphene/h-BN superlattices with arbitrary defects included via Green's function  methods. Awarded Tessella Prize for software development.}

\cvitem{UG}{\textbf{Interactions of quantised vortices in superfluid helium}}
\cvitem{}{with Dr Paul Walmsley \& Prof Andrei Golov \emph{(University of Manchester)}}
%\cvitem{}{Spent two summers developing \colourhref{https://github.com/ml-evs/vfmcpp}{\texttt{vfmcpp}}, a C++/OpenMP implementation of the vortex filament model of superfluid helium, to study microscopic vortex dynamics and reconnection events \cite{PhysRevFluids.1.044502}. }

\cvitem{UG}{\textbf{Hard sphere packing of nanotube-encapsulated fullerenes}}
\cvitem{}{with Dr Ho-Kei Chan \& Prof Elena Besley \emph{(University of Nottingham)}}
%\cvitem{}{Application of a novel hard sphere packing regime to study CNT-encapsulated fullerenes.}
\section{\texttt{COMPUTING}}
\centering Exposure: \textbf{Daily}, Intermittent, \textit{Occasional}.\\[0.2em]
\cvdoubleitem{Languages}{\textbf{Python}, Fortran, Cython, C++}{Databases}{\textbf{MongoDB}, \textit{SQL}}
\cvdoubleitem{DFT}{\textbf{CASTEP}, Quantum Espresso}{Packages}{\textbf{NumPy, SciPy, spglib}, scikit-learn}
\cvdoubleitem{Platforms}{\textbf{Linux}, *nix}{Practices}{\textbf{Test-driven development, CI}}
\cvdoubleitem{Data viz}{\textbf{matplotlib}, Bokeh, d3.js}{Tools}{\textbf{git}, \textbf{vim}, \textbf{Docker}}

\section{\texttt{(TEACHING + SERVICE)}}
\cvitem{2018--}{Reviewed manuscripts for \emph{Scientific Reports}}
\cvitem{2016--}{Active member of TCM \texttt{sysadmin} team, Cavendish Laboratory} 
\cvitem{2019}{Demonstrator: Part II Computational Physics, Cavendish Laboratory
\begin{itemize}
    \item[--] Demonstrated scientific Python to beginners in weekly labs.
    \item[--] Wrote and delivered a tutorial on the basics of
        \colourhref{https://github.com/ml-evs/part2-computing-git-tutorial}{version control
        with Git}.
\end{itemize}}
\vspace{-0.2in}
\cvitem{2018}{Demonstrator: Graduate-level Atomistic Modelling of Materials, Cavendish Laboratory}
\cvitem{2016--2018}{Supervisor: Part IB Electromagnetism, Dynamics and Thermodynamics, Selwyn College
\begin{itemize}
    \item[--] Small group teaching, providing detailed feedback on assigned problems.
\end{itemize}}
\cvitem{2016--2018}{Demonstrator: Part IB Computational Physics (C++), Cavendish Laboratory}
\cvitem{2016--}{Demonstrator: 3x at annual CASTEP workshop, University of Oxford}
\cvitem{2017}{Volunteer: 2nd Conference of Research Software Engineers, University of Manchester}
\cvitem{2016--2017}{Volunteer: Key Stage 2 Code Club, Ridgefield Primary School, Cambridge}
\cvitem{2016}{Demonstrator: Graduate-level Electronic Structure, Cavendish Laboratory}
\cvitem{2012--2015}{{Tutor: GCSE Maths \& Key Stage 2 Programming for \colourhref{http://thetutortrust.org/}{The Tutor Trust}, Manchester}
\begin{itemize}
    \item[--] Provided tuition to small groups and `looked after children' across 15 schools.
    \item[--] Helped lead a successful pilot to teach primary school children programming using Scratch.\end{itemize}}
\vspace{-0.3in}

\section{\texttt{PROFESSIONAL ACTIVITIES}}
\cvitem{2015--}{Presented posters and talks at 20 conferences and workshops both
nationally and internationally. Selected works can be found on my
\colourhref{http://ml-evs.github.io/talks.html}{personal website}.}
\cvitem{2018}{Contributed talk: \emph{Sn-P anodes for potassium-ion batteries: insights from crystal structure prediction}, SMARTER6 Conference, Ljubljana, Slovenia}
\cvitem{}{Invited talk: \emph{\texttt{matador}: databases and crystal structure prediction} (\colourhref{http://www.tcm.phy.cam.ac.uk/~me388/optimade/matador\_optimade.pdf}{slides}), CECAM Workshop, Open Databases Integration for Materials Design, EPFL, Switzerland}
\cvitem{2017}{Invited talk: \emph{Crystal structure prediction for next-generation battery anodes} (\colourhref{http://www.tcm.phy.cam.ac.uk/~me388/ss\_11.17/}{slides}), Solid State Seminar Series, University of Cambridge}
\cvitem{}{Poster Presentation: 13th RSC Conference in Materials Chemistry (\colourhref{http://www.tcm.phy.cam.ac.uk/~me388/posters/mc13.pdf}{poster}), University of Liverpool}
\cvitem{2016}{Poster Presentation: SMARTER5 Conference, University of Bayreuth, Germany}

\section{\texttt{(AWARDS + HONOURS)}}
\cvitem{2018}{Tier-2 HPC Resource Allocation: PI on project awarded 4 MCPUh, \emph{Crystal structure prediction for next-generation solar absorbers}, \textbf{M. L. Evans}, D. O. Scanlon and A. J. Morris.}
\cvitem{}{HPC Midlands+ Substantial Project: awarded 1.3 MCPUh for \emph{High-throughput materials discovery for energy applications}, \textbf{M. L. Evans} and A. J. Morris.}
\cvitem{2017}{Tier-2 HPC Resource Allocation: Co-investigator on project awarded 4 MCPUh, \emph{Ab initio structure prediction for next-generation battery materials}, B. Karasulu, \textbf{M. L. Evans} and A. J. Morris.}
\cvitem{2015}{Tesella Prize for Software, University of Manchester, for the most effective use of software in a final year physics project.}
\cvitem{2013, 2014}{Undergraduate research bursary for two summers as an undergraduate, totalling £4200.}
\cvitem{2011--2015}{Means-tested and merit based scholarship to study at the University of Manchester, worth £12,000.}

\section{\texttt{PUBLICATIONS}}
%Status: {\color{darkgray} in preparation}, {\color{OliveGreen} preprint}, {published}.
\nocite{*}
{\color{darkgray}\printbibliography[heading=none,keyword=inprep]} 
\nocite{*}
\vspace{-0.25in}
\printbibliography[heading=none,notkeyword=inprep]


%\section{\texttt{(CONFERENCES + PRESENTATIONS)}}
%\cvitem{2018}{\emph{Sn-P anodes for potassium-ion batteries}, Poster, Thomas Young Center 5th Energy Materials Workshop, UCL}
%\cvitem{}{\emph{Sn-P anodes for potassium ion batteries}, Poster and Pico Talk, CCP9 Young Researchers Event, York}
%\cvitem{}{\emph{\texttt{matador}: databases and crystal structure prediction} (\colourhref{http://www.tcm.phy.cam.ac.uk/~me388/optimade/matador_optimade.pdf}{slides}), Invited Talk, OPTiMaDe workshop, CECAM@EPFL, Switzerland}
%\cvitem{}{Total Energy and Force Methods, Poster Presentation, University of Cambridge}
%\cvitem{2017}{\emph{Crystal structure prediction for next-generation battery anodes} (\colourhref{http://www.tcm.phy.cam.ac.uk/~me388/ss_11.17/}{slides}), Invited Talk, Solid State Seminar Series, University of Cambridge}
%\cvitem{}{Second conference of Research Software Engineers, Volunteer, University of Manchester}
%\cvitem{}{CASTEP Developer Workshop, Demonstrator and Poster Presentation, University of Oxford}
%\cvitem{}{13th RSC Conference in Materials Chemistry, Poster Presentation (\colourhref{http://www.tcm.phy.cam.ac.uk/~me388/posters/mc13.pdf}{link}), University of Liverpool}
%\cvitem{}{STFC Annual Battery Meeting, Attendee, Abingdon}
%\cvitem{}{CCP9 Young Researchers Event, Poster Presentation, University of Cambridge}
%\cvitem{}{Scientific Computing Day, Poster Presentation, University of Cambridge}
%\cvitem{2016}{High Performance Computing Autumn Academy, Presenter, University of Cambridge}
%\cvitem{}{SMARTER5, Poster Presentation, University of Bayreuth, Germany}
%\cvitem{}{CASTEP Workshop, Demonstrator and Poster Presentation, University of Oxford}
%\cvitem{}{CCP9 Young Researchers Event, Poster Presentation, University of York}

\section{\texttt{REFEREES}}
%\cvitem{}{References available on request.}
\cvitem{}{Dr Andrew Morris, University of Birmingham; \colourhref{mailto:ajm255@cam.ac.uk}{\texttt{a.j.morris.1@bham.ac.uk}}}
\cvitem{}{Prof Mike Payne, University of Cambridge; \colourhref{mailto:mcp1@cam.ac.uk}{\texttt{mcp1@cam.ac.uk}}}
\cvitem{}{Dr Paul Walmsley, University of Manchester; \colourhref{mailto:paul.walmsley@manchester.ac.uk}{\texttt{paul.walmsley@manchester.ac.uk}}}

\pagestyle{fancy}
%\pagestyle{empty}

\end{document}
