\PassOptionsToPackage{dvipsnames}{xcolor}
\documentclass[11pt,a4paper,sans]{moderncv}        % possible options include font size ('10pt', '11pt' and '12pt'), paper size ('a4paper', 'letterpaper', 'a5paper', 'legalpaper', 'executivepaper' and 'landscape') and font family ('sans' and 'roman')

\usepackage{fontspec}
\usepackage{soul}
\setsansfont[BoldFont={Archivo Narrow Bold}]{Archivo Narrow}
\setmonofont[Scale=0.9, BoldFont={Iosevka SS07}]{Iosevka SS07}
% moderncv themes
\moderncvstyle{classic}                             % style options are 'casual' (default), 'classic', 'banking', 'oldstyle' and 'fancy'
\moderncvcolor{blue}                               % color options 'black', 'blue' (default), 'burgundy', 'green', 'grey', 'orange', 'purple' and 'red'
\definecolor{niceblue}{RGB}{139, 233, 253}
\definecolor{mypurple}{RGB}{189,147,249}
\definecolor{red}{rgb}{0, 1, 0}
\definecolor{niceyellow}{HTML}{e0af68}
\definecolor{nicegreen}{HTML}{565f89}
\colorlet{color1}{mypurple}
\colorlet{color2}{nicegreen}
%\definecolor{bg}{HTML}{24283b}
\definecolor{bg}{HTML}{1a1b26}
%\definecolor{fg}{RGB}{248,248,242}
\definecolor{fg}{HTML}{a9b1d6}
\colorlet{color0}{nicegreen}
\colorlet{preprintcolour}{nicegreen}
\definecolor{darkgray}{gray}{0.4}
\usepackage[unicode,pdfencoding=auto]{hyperref}
%\definecolor{color2}{RGB}{56,115,178}
%\usepackage{doi}
\usepackage[style=nature,sorting=ymdnt,defernumbers=false,date=year,sortcites=false,maxnames=5,minnames=3,maxbibnames=5]{biblatex}
\addbibresource{pubs.bib}

%\DeclareSourcemap{
%  \maps[datatype=bibtex]{
%    \map{
%      \step[fieldsource=keywords, match=inprep,final,preprint]
%      \step[fieldset=keywords, fieldvalue=inprep]
%    }
%  }
%}

\DeclareSortingScheme{ymdnt}{
    \sort[direction=descending]{
        \field{year}
    }
    \sort[direction=descending]{
        \field{month}
        \literal{99999}
    }
}

%hyphenation
\tolerance=1
\emergencystretch=\maxdimen
\hyphenpenalty=10000
\hbadness=10000

%\usepackage[-1]{pagesel}

% Reverse numbering in publications list
\newcounter{entrycount}
\AtDataInput{\stepcounter{entrycount}}
\DeclareFieldFormat{labelnumber}{\mkrevbibnum{#1}}
\newcommand{\mkrevbibnum}[1]{\number\numexpr\value{entrycount}+1-#1}

\newcommand{\colourhref}[2]{\setulcolor{red}\href{#1}{\underline{#2}}}
\newcommand{\greencolourhref}[3][preprintcolour]{\href{#2}{\color{#1}#3}}
\newcommand{\doi}[1]{{\href{https://doi.org/#1}{{#1}}}}
\newcommand{\eprint}[1]{{\greencolourhref{https://doi.org/#1}{{#1}}}}
%\renewcommand{\familydefault}{\sfdefault}         % to set the default font; use '\sfdefault' for the default sans serif font, '\rmdefault' for the default roman one, or any tex font name
\nopagenumbers{}                                  % uncomment to suppress automatic page numbering for CVs longer than one page

%\cfoot{\emph{Last modified: \today}}

\usepackage{etoolbox}
\makeatletter
\patchcmd{\makecvtitle}% <cmd>
  {\httplink{\@homepage}}% <search>
  {{\ifx\@homepage@shorthand\relax
     \httplink{\@homepage}% Used \homepage{<URL>}
   \else
     \httplink[\@homepage@shorthand]{\@homepage}% Used \homepage[<desc>]{<URL>}
   \fi}}% <replace>
  {}{}% <succes><failure>
\patchcmd{\thebibliography}
{\advance\leftmargin\labelsep}
  {\labelsep=0.7cm \advance\leftmargin\labelsep}{}{}
\RenewDocumentCommand{\homepage}{o m}{%
  \let\@homepage@shorthand\relax%
  \providecommand\@homepage{#2}%
  \IfNoValueF{#1}{\def\@homepage@shorthand{#1}}%
}
\makeatother


% adjust the page margins
\usepackage[scale=0.8]{geometry}
%\usepackage[paperheight=44in,margin=1in,scale=0.8]{geometry}
%\setlength{\hintscolumnwidth}{3cm}                % if you want to change the width of the column with the dates
%\setlength{\makecvheadnamewidth}{12cm}            % for the 'classic' style, if you want to force the width allocated to your name and avoid line breaks. be careful though, the length is normally calculated to avoid any overlap with your personal info; use this at your own typographical risks...

% personal data
\name{Matthew}{Evans}
%\title{Resumé title}                               % optional, remove / comment the line if not wanted
\email{matthew@ml-evs.science}                               % optional, remove / comment the line if not wanted
\homepage{ml-evs.science}
\social[github]{ml-evs}                           % optional, remove / comment the line if not wanted
\quote{
\texttt{decentralized data management \textbullet\, open science \& software
\texttt{materials discovery \textbullet\, ab initio calculations
}}}

\renewcommand*{\bibliographyitemlabel}{[\arabic{enumiv}]}

\begin{document}
%\pagecolor{bg}
%\color{fg}
%-----       resume       ---------------------------------------------------------
%\input{cover_letter.tex}
%\newpage
\makecvtitle

\section{\texttt{RESEARCH INTERESTS}}
\cvitem{}{My background in computational materials science has left me with an overarching interest in the application of machine learning, open source software \& infrastructure, and data management practices to accelerate and enhance scientific workflows for discovery in the chemical and materials sciences.}
%Most recently, I have been leading the development, deployment and proliferation of \colourhref{https://github.com/the-grey-group/datalab}{datalab}, a decentralized data management platform for managing samples and devices in the lab. In my reserach fellowship, I work on high-throughput and machine-learning-accelerated workflows for materials discovery and design. I am also a leading developer in the \colourhref{https://optimade.org}{OPTIMADE} consortium, which aims to make materials databases interoperable to enable data aggregation for materials discovery. My PhD research focused on performing \emph{ab initio} crystal structure prediction for beyond-Li battery electrode materials, in close collaboration with experimentalists.}
%\pagestyle{empty}
\section{\texttt{EDUCATION}}
\cventry{2016--2023}{PhD Physics}{(submitted July 2023)}{Theory of Condensed Matter Group, University of Cambridge}{}{}  % arguments 3 to 6 can be left empty
\cventry{2015--2016}{MPhil Scientific Computing}{}{University of Cambridge}{\textit{Pass with distinction}}{}  % arguments 3 to 6 can be left empty
\cventry{2011--2015}{MPhys Physics with Theoretical Physics}{}{University of Manchester}{\textit{First Class (Hons)}}{}

\section{\texttt{SELECTED EXPERIENCE}}
\cvitem{2020--}{\textbf{Research Assistant} then \textbf{BEWARE Research Fellow} (2022 onwards)}
\cvitem{}{Universit\`{e} catholique de Louvain and Matgenix, with Prof Gian-Marco Rignanese}
\cvitem{}{
\begin{itemize}
    \item[--] Co-creator and architect of \colourhref{https://github.com/datalab-org/datalab}{\emph{datalab}}, open source data management software for sample tracking and characterisation, lab management, and machine learning, deployed at several labs internationally.
    \item[--] High-throughput machine-learning accelerated workflows for materials discovery and design.
    \item[--] Leading development of the \colourhref{https://optimade.org}{OPTIMADE} API specification and associated software.
\end{itemize}}
\vspace{-0.2in}
\cvitem{2021--}{\textbf{Visiting Researcher: Data management platforms for materials chemistry research}}
\cvitem{}{University of Cambridge, with Prof Clare Grey FRS}
\cvitem{}{
\begin{itemize}
    \item[--] Developing bespoke data management platforms for materials chemistry and battery research.
    \item[--] Supervising contributions from a full-time software developer and providing user training.
\end{itemize}}
\vspace{-0.2in}
\cvitem{2024--}{\textbf{Scientific Software Consultant} and \textbf{Director} (part-time)}
\cvitem{}{\colourhref{https://datalab.industries}{datalab industries ltd.}}
\cvitem{}{
\begin{itemize}
    \item[--] Supporting the open source development of OPTIMADE and \emph{datalab} via consultancy services.
    \item[--] Customisation and deployment of \emph{datalab} for industrial R\&D and academic labs.
\end{itemize}}
\vspace{-0.2in}
\cvitem{2016--2020}{\textbf{PhD student: Crystal structure prediction for next-generation energy storage}}
\cvitem{}{University of Cambridge, with Dr Andrew Morris}
\cvitem{}{\begin{itemize}
    \item[--] Computional materials discovery for conversion anodes for Li, Na and K-ion batteries.
    \item[--] Author of two open-source Python packages: database approaches for high-throughput calculations and materials design with \colourhref{http://matador.science}{\texttt{matador}} and crystal structure prediction with \colourhref{http://ilustrado.readthedocs.io}{\texttt{ilustrado}}.
    %\item[--] Active member of the \colourhref{https://optimade.org}{OPTIMADE consortium} for materials database interoperability and author of the \colourhref{https://github.com/Materials-Consortia/optimade-python-tools}{\texttt{optimade-python-tools}} package (used in production by The Materials Project, NOMAD, Materials Cloud and others) and \colourhref{https://odbx.science}{\texttt{odbx}} implementation.
\end{itemize}}
%PhD research undertaken as a member of the EPSRC CDT for Computational Methods in Materials Science.
\vspace{-0.2in}
\section{\texttt{COMPUTING}}
%\pagestyle{empty}
%\centering Exposure: \textbf{Daily}, Intermittent, \textit{Occasional}.\\[0.2em]
\cvdoubleitem{Languages}{\textbf{Python}, Javascript, Vue.js, Fortran, C++}{Tools}{\textbf{git}, \textbf{vim}, \textbf{Docker}, \textbf{Ansible}, \textbf{Terraform}}
%\cvdoubleitem{DFT}{\textbf{CASTEP}, Quantum Espresso, \emph{GPAW}}{Stack}{FastAPI, pydantic, Flask, Tensorflow}
\cvdoubleitem{Practices}{\textbf{Test-driven development, CI/CD, Cloud Automation, HPC}}{Expertise}{Web APIs, databases, machine learning, high-throughput workflows}
\clearpage
\section{\texttt{OTHER EXPERIENCE}}
%\cvitem{2022}{\textbf{Postdoctoral Research Associate: Recommender systems}}
%\cvitem{}{\emph{Cambridge Crystallographic Data Centre}}
\cvitem{2022}{\textbf{Postdoctoral Researcher: Recommender systems for crystal structure search}}
\cvitem{}{Cambridge Crystallographic Data Centre, Cambridge, UK}
\cvitem{2019}{\textbf{Visiting Researcher: Machine learning for materials discovery}}
\cvitem{}{Aalto University, Finland, with Profs Adam Foster \& Patrick Rinke}
\cvitem{2019}{\textbf{Scientific Software Developer (Intern): Multi-objective optimisation}}
\cvitem{}{\emph{Enthought Inc., Cambridge}}
\cvitem{2014, 2015}{\textbf{UG research: Interactions of quantised vortices in superfluid helium}}
\cvitem{}{University of Manchester, with Dr Paul Walmsley \& Prof Andrei Golov}
%\cvitem{2014--2015}{\textbf{MPhys project: Electronic structure of defects in graphene superlattices}}
%\cvitem{}{University of Manchester, with Prof Francisco Guinea}
\cvitem{2013}{\textbf{UG research: Hard sphere packing of nanotube-encapsulated fullerenes}}
\cvitem{}{University of Nottingham, with Dr Ho-Kei Chan \& Prof Elena Besley}
%\cvitem{}{Application of a novel hard sphere packing regime to study CNT-encapsulated fullerenes.}
%\vspace{-0.2in}
%\cvdoubleitem{Data viz}{\textbf{matplotlib}, Bokeh, d3.js}{Tools}{\textbf{git}, \textbf{vim}, \textit{Docker}}
%\clearpage
\section{\texttt{SELECTED (AWARDS + HONOURS)}}
\cvitem{2022}{BEWARE2 Fellowship from the Wallonia-Brussels Federation (approx. €300,000).}
\cvitem{2021}{PI for ``Interoperable data management for fundamental battery research'', BIG-MAP External Stakeholder Initiative (approx. €150,000).}
%\cvitem{2019}{HPC-Europa 3 funding to visit Aalto University for 7 weeks and associated computing time.}
%\cvitem{2018}{Tier-2 HPC Resource Allocation: PI on project awarded 2 MCPUh, \emph{Crystal structure prediction for next-generation solar absorbers}, \textbf{M. L. Evans}, D. O. Scanlon and A. J. Morris.}
%\cvitem{}{HPC Midlands+ Substantial Project: awarded 1.3 MCPUh for \emph{High-throughput materials discovery for energy applications}, \textbf{M. L. Evans} and A. J. Morris.}
%\cvitem{2017}{Tier-2 HPC Resource Allocation: Co-investigator on project awarded 4 MCPUh, \emph{Ab initio structure prediction for next-generation battery materials}, B. Karasulu, \textbf{M. L. Evans} and A. J. Morris.}
%\cvitem{2015}{Tesella Prize for Software, University of Manchester, for the most effective use of software in a final year physics project.}
%\cvitem{2013, 2014}{Undergraduate research bursary for two summers as an undergraduate, totalling £4200.}
%\cvitem{2011--2015}{Means-tested and merit based scholarship to study at the University of Manchester, worth £12,000.}
%\clearpage

\section{\texttt{SELECTED (TEACHING + SERVICE)}}
\cvitem{2018--}{Reviewed manuscripts and data for \emph{JOSS} (x6), \emph{Digital Discovery} (x5), \emph{J. Phys.: Cond. Mat.} (x4), \emph{Mach. Learn.: Sci. Technol.} (x2) (IOP Outstanding Reviewer 2024), \emph{npj. Comp. Mater.} (x1), \emph{Sci Data} (x1), \emph{Sci. Rep.} (x1)}
\cvitem{2022--2024}{Proposed and co-lead a \colourhref{https://www.marda-alliance.org/}{MaRDA} working group on \colourhref{https://github.com/marda-alliance/metadata_extractors}{metadata extractors} for materials science.}
%\cvitem{2020--}{Co-chair of the Research Data Alliance (RDA) IG \emph{Materials Data, Infrastructure \& Interoperability}}
\cvitem{2022--2024}{Initiator and organiser of the CECAM Workshop series \emph{Machine-actionable Data Interoperability for Chemical Sciences} (\colourhref{https://madices.github.io/}{MADICES}, February 2022 and April 2024)}
%\cvitem{2021}{Lecturer for ``Working with Materials Databases'' at the ICTP-East African Institute for Fundamental Research Training School \emph{Machine Learning for Electronic Structure and Molecular Dynamics}}
%\cvitem{2021}{Mentor at Acceleration Consortium Hackathon on Scientific Databases}
%\cvitem{2021}{Developed and delivered a \colourhref{https://github.com/Materials-Consortia/optimade-tutorial-exercises}{2-day OPTIMADE tutorial} for the NOMAD Virtual Tutorial Series.}
%\cvitem{2016--2020}{Active member of TCM \texttt{sysadmin} team, Cavendish Laboratory}
\cvitem{2019-2021}{Demonstrator: Part II Computational Physics, 3x Part IB Intro to Computing, Cavendish Laboratory}
%\begin{itemize}
%    \item[--] Demonstrated scientific Python to beginners in weekly labs (2019 only).
%    \item[--] Conceptualised and delivered a tutorial on the basics of
%        \colourhref{https://github.com/ml-evs/git-tutorial}{version control
        %with Git} (2019-2021).
%\end{itemize}}
%\vspace{-0.2in}
%\cvitem{2018--2019}{Demonstrator: 2x Graduate-level Atomistic Modelling of Materials, Cavendish Laboratory}
\cvitem{2016--2018}{Supervisor: 2x Part IB Electromagnetism, Dynamics and Thermodynamics, Selwyn College}
%\begin{itemize}
    %\item[--] Small group teaching, providing detailed feedback on assigned problems.
%\end{itemize}}
%\vspace{-0.2in}
%\cvitem{2016--2019}{Demonstrator: 3x Part IB Introduction to Computing (C++), Cavendish Laboratory}
%\cvitem{2016--2019}{Demonstrator: 4x at annual CASTEP workshop, University of Oxford}
%\cvitem{2017}{Volunteer: 2nd Conference of Research Software Engineers, University of Manchester}
%\cvitem{2016--2017}{Volunteer: Key Stage 2 Code Club, Ridgefield Primary School, Cambridge}
%\cvitem{2016}{Demonstrator: Graduate-level Electronic Structure, Cavendish Laboratory}
\cvitem{2012--2015}{{Tutor: GCSE Maths \& Key Stage 2 Programming for \colourhref{https://thetutortrust.org/}{The Tutor Trust}, Manchester}}
%\begin{itemize}
    %\item[--] Provided tuition to small groups and `looked after children' across 15 schools.
    %\item[--] Helped lead a successful pilot to teach primary school children programming using Scratch.\end{itemize}}
%\vspace{-0.3in}

\section{\texttt{SELECTED RECENT PRESENTATIONS}}
\cvitem{2025}{Invited seminar: \emph{Decentralized materials research data management, curation and dissemination for accelerated discovery}, PSDI Polymer Data Workshop, Loughborough University}
\cvitem{}{Invited seminar: \emph{Decentralized materials research data management, curation and dissemination for accelerated discovery}, Computational Chemistry Seminar Series, University of Warwick}
\cvitem{2024}{Invited talk: \emph{Decentralized materials research data management, curation and dissemination for accelerated discovery}, Democratizing AI in Materials Science —- A Pathway to Broaden the Impact of Materials Research, MRS Fall Meeting, Boston, USA}
\cvitem{}{Contributed talk/paper: \emph{Optical materials discovery and design with federated databases and machine learning}, Faraday Discussions, University of Oxford, United Kingdom.}
\cvitem{}{Invited talk: \emph{Federated, interoperable databases for accelerated materials discovery and design}, CECAM Flagship Workshop on MLIPs and Accessible Databases, Grenoble, France.}
%\cvitem{2023}{Invited seminar: \emph{Interoperable data management for fundamental materials chemistry research}, Department of Chemistry, University of Nottingham, United Kingdom.}
%\cvitem{2023}{Contributed talk: \emph{Interoperable data management for fundamental battery research}, RSC Annual Advanced Battery Materials Symposium, Institute of Physics, United Kingdom.}
%\cvitem{}{Invited seminar: \emph{Interoperable data management for fundamental battery research}, Conductivity and Catalysis Lab, Technische Universität Berlin, Germany.}
\cvitem{}{Invited talk: \emph{Open Databases Integration for Materials Design} at the CECAM Flagship Workshop for FAIR and TRUE Soft Matter Simulations, Max Planck Institute for Polymer Research, Germany.}
%\cvitem{}{Invited seminar: \emph{Interoperable data management for fundamental battery research}, Laboratory of Materials Simulation, Paul Scherrer Institut, Switzerland.}
%\cvitem{}{Invited talk: \emph{Metadata extractors for interoperable ETL}, MaRDA Alliance Annual Meeting}
\cvitem{}{Invited talk: \emph{Open Databases Integration for Materials Design} at the Actively Learning Materials Science (AL4MS2023) workshop, Aalto University, Finland.}
%\cvitem{2021}{Invited panel discussions: \emph{International Materials Data: Joint Meeting} and \emph{Metadata for Data Indexing and Discovery in Materials Science}, Research Data Alliance (RDA) 18th Virtual Plenary Meeting}
%\cvitem{}{Invited talk: \emph{The OPTIMADE Ecosystem}, DoE Battery Genome Initiative}
%\cvitem{}{Invited panel discussion: \emph{Delivery platforms for open marketplaces}, Research Data Alliance (RDA) 17th Virtual Plenary Meeting}
%\cvitem{2020}{Invited talk: \emph{The OPTIMADE Specification}, Research Data Alliance (RDA) 16th Virtual Plenary Meeting: Data Infrastructure for Collaborations in Materials Research}
%\cvitem{}{Invited talk and workshop demonstration: \emph{\texttt{odbx \& OPTIMADE}} and \emph{\texttt{optimade-python-tools}}, CECAM Workshop, Open Databases Integration for Materials Design 2020}
%\cvitem{2019}{Contributed talk: \emph{Phosphorus anodes for potassium-ion batteries: insights from crystal structure prediction}, EMRS Spring 2019, Nice, France}
%\cvitem{}{Invited talk: \emph{\texttt{matador \& OPTIMADE}}, CECAM Workshop, Open Databases Integration for Materials Design 2019, EPFL, Switzerland}
%\cvitem{2018}{Contributed talk: \emph{Sn-P anodes for potassium-ion batteries: insights from crystal structure prediction}, SMARTER6 Conference, Ljubljana, Slovenia}
%\cvitem{}{Invited talk: \emph{\texttt{matador}: databases and crystal structure prediction} (\colourhref{http://www.tcm.phy.cam.ac.uk/~me388/optimade/matador\_optimade.pdf}{slides}), CECAM Workshop, Open Databases Integration for Materials Design 2018, EPFL, Switzerland}
%\cvitem{2017}{Invited seminar: \emph{Crystal structure prediction for next-generation battery anodes}, Solid State Seminar Series, University of Cambridge}
%\cvitem{}{Poster Presentation: 13th RSC Conference in Materials Chemistry (\colourhref{http://www.tcm.phy.cam.ac.uk/~me388/posters/mc13.pdf}{poster}), University of Liverpool}
%\cvitem{2016}{Poster Presentation: SMARTER5 Conference, University of Bayreuth, Germany}
%\pagebreak

\clearpage
\section{\texttt{PUBLICATIONS}}
Underline indicates (joint) first authorship (reordered where appropriate). Full list with OA links available online (\colourhref{https://ml-evs.science/papers}{\texttt{https://ml-evs.science/papers}}, \colourhref{https://orcid.org/0000-0002-1182-9098}{ORCiD}, \colourhref{https://scholar.google.co.uk/citations?user=3aG55qYAAAAJ&hl=en}{Google Scholar}).
%\nocite{*}
%{\color{darkgray}\printbibliography[heading=none,keyword=inprep]}
\nocite{*}
\printbibliography[heading=none, keyword=selected]

%\section{\texttt{(CONFERENCES + PRESENTATIONS)}}
%\cvitem{2018}{\emph{Sn-P anodes for potassium-ion batteries}, Poster, Thomas Young Center 5th Energy Materials Workshop, UCL}
%\cvitem{}{\emph{Sn-P anodes for potassium ion batteries}, Poster and Pico Talk, CCP9 Young Researchers Event, York}
%\cvitem{}{\emph{\texttt{matador}: databases and crystal structure prediction} (\colourhref{http://www.tcm.phy.cam.ac.uk/~me388/optimade/matador_optimade.pdf}{slides}), Invited Talk, OPTIMADE workshop, CECAM@EPFL, Switzerland}
%\cvitem{}{Total Energy and Force Methods, Poster Presentation, University of Cambridge}
%\cvitem{2017}{\emph{Crystal structure prediction for next-generation battery anodes} (\colourhref{http://www.tcm.phy.cam.ac.uk/~me388/ss_11.17/}{slides}), Invited Talk, Solid State Seminar Series, University of Cambridge}
%\cvitem{}{Second conference of Research Software Engineers, Volunteer, University of Manchester}
%\cvitem{}{CASTEP Developer Workshop, Demonstrator and Poster Presentation, University of Oxford}
%\cvitem{}{13th RSC Conference in Materials Chemistry, Poster Presentation (\colourhref{http://www.tcm.phy.cam.ac.uk/~me388/posters/mc13.pdf}{link}), University of Liverpool}
%\cvitem{}{STFC Annual Battery Meeting, Attendee, Abingdon}
%\cvitem{}{CCP9 Young Researchers Event, Poster Presentation, University of Cambridge}
%\cvitem{}{Scientific Computing Day, Poster Presentation, University of Cambridge}
%\cvitem{2016}{High Performance Computing Autumn Academy, Presenter, University of Cambridge}
%\cvitem{}{SMARTER5, Poster Presentation, University of Bayreuth, Germany}
%\cvitem{}{CASTEP Workshop, Demonstrator and Poster Presentation, University of Oxford}
%\cvitem{}{CCP9 Young Researchers Event, Poster Presentation, University of York}

\pagestyle{fancy}
%\pagestyle{empty}

\end{document}
